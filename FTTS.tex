\section{Timed Rebeca and Floating Time Transition System} \label{sec::FTTS}
Active objects and actors are encapsulated modules with no shared variables. We also choose to have atomic execution of message servers (i.e. methods, event handlers) which gives us a macro-step semantics and models a non-preemptive execution of the handlers.
Our actors are reactive, when sending a message they are not blocked and there is no explicit receive. So, there is no coupling via shared variables, no coupling because of waiting for another actor to return a value for a remote procedure call, and no coupling because of a context dependency caused by having a future  construct in the language.
This isolation helps in more efficient analysis, and reduces the state space.
Moreover, if we are only interested in event-based properties we may be able to abstract even more and just keep the states that are followed by a transition which we are interested in. This type of reduction is not straight forward as we need to prove that we are preserving the order of the events while abstracting away some of the states and transitions. This is what is done in partial order reduction.

Floating Time Transition System (FTTS) is proposed based on the above discussion. The idea behind FTTS is similar to partial order reduction (POR) but the technique does not fit exactly in the definition of POR. What we mean by floating time is that in each state of the state space, different actors do not necessarily have the same local clock, i.e.,  actors are not synchronised on their local time in the state space. We consider this as letting the time \textit{float} across the actors in the state space. 
To avoid confusion, it is important to note the different models in different levels of abstraction, and also layering of models. We have (1) distributed systems, we use (2) Timed Rebeca to model distributed systems, and we model (3) the state space as Floating Time Transition System to do the analysis. 
%
Note that at the level of Timed Rebeca, actors have synchronised local clocks which gives us a notion of global time across the model. We use time stamps, and time stamps are comparable across all actors in the model. This makes our model simpler and more understandable, and our analysis more efficient.
But in distributed systems we cannot assume synchronised clocks and time stamps. 
For that assumption to be valid and faithful enough to the system,  we rely on layering and different responsibilities for different layers. For distributed actors to be able to have synchronised clocks and comparable time stamps we rely on the lower-level network protocols to provide that for us. 

In Timed Rebeca we have a concept of time and we can consider that each statement is executed at a certain point in time. Note that we are now talking at the level of the Rebeca model, the notion of time is the model time, and we do not need to worry about synchronising the clocks among different components in the distributed system. We assume that local clocks of actors are synchronised and we can have comparable time stamps on each statement in the actors.
%
In Timed Rebeca models, we use a \texttt{delay(t)} statement to show the computation delay. Other statements are assumed to be executed in zero time. We use  \texttt{after(t)} in combination with a \texttt{send} message statement, it means that the time stamp of the message when it is put in the queue of the receiver is the value of the local clock of the sender (\texttt{now} in the sender) plus the value of \texttt{t}.
The progress of time is forced by the \texttt{delay} statement and also by \texttt{after}. 
We can assume that the time stamp of all the statements are zero when a model starts to execute, then in each actor the local time is increased by value of \texttt{t} if there is a \texttt{delay(t)} statement.
A \texttt{send} statement with an  \texttt{after} does not cause any increase in the local time per se. The statement following the \texttt{send} statement has the same time stamp as the \texttt{send} statement itself.
The \texttt{after} construct can cause an increase in the time when the actor picks the message annotated by \texttt{after} to be executed. The local time of the receiver actor is set to the time stamp of the message, unless it is already greater than that.
The latter situation means that the message has been sitting in the queue while the actor has been busy executing another message. Remember that messages are executed atomically and are not preempted.
The \texttt{after} construct can be used to model the network delay, and also to model periodic events.


If we use the standard Timed Transition System to generate the state space for Timed Rebeca model we distinguish three types of transitions: $\tau$ transitions, \textit{events}, and \textit{timed} transitions.
In FTTS we reduce that to only \textit{events} transitions.
%





