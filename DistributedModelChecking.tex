\section{Distributed Model Checking} \label{sec::DMC}
In addition to benefiting from the isolation of actors for reducing the size of state spaces, this property can be used for more efficient analysis of huge state spaces. A major limiting factor in applying model checking for the analysis of real-world systems is the huge amount of space and time required to store and explore state spaces. Distributed model checking is a technique for analyzing these types of state spaces; in which, state spaces are partitioned into some slices and each slice is assigned to a computational node to be analyzed. The efficiency of this technique depends on the communication costs among computational nodes which is related to the distribution policy of states among nodes \cite{DBLP:journals/entcs/OrzanPE05}. Another, more fine-grained, representative of communication cost is the number of split transitions; a split transition is a transition between two states,  where the hosts of source and destination states are different nodes. In \cite{DBLP:journals/eceasst/KhamespanahSMSR15} we showed how the actor model can be used to reduce teh number of split transitions. We introduced a new state distribution policy based on the so-called Call Dependency Graph (CDG) of actor models. A CDG represents the abstract causality relation among messages of actors. Our abstraction is akin to the dynamic representation of actor's event activation causality proposed by Clinger \cite{clinger}. 

The most primitive and widely used distribution policy is random state distribution \cite{DBLP:journals/entcs/GaravelMS13}. Random state distribution policy distributes states among nodes based on their hash values. Random distribution policy guarantees load balancing. However, it is not an effective technique as cycles are scattered over many different nodes. In \cite{DBLP:journals/entcs/OrzanPE05}, another state space distribution policy is suggested to improve the locality of cycles. This policy is based on the static analysis of an abstracted model and detects \emph{may} or \emph{must} transition relations among states \cite{DBLP:conf/lics/LarsenT88}. Based on this analysis, if two states have a \emph{must} relation, they should be stored in a same node. We use a similar idea in our state distribution policy and show that using the CDG improves the locality of cycles by reducing the split transitions in the state space. In other words, we find the \emph{must} relations among the states of actor models using the CDG. Our technique is applicable to other service-oriented models where the unit of concurrency can be modeled as isolated autonomous active objects and message passing is the only way of communication. 

Clinger's event diagram comprise vertices (called \emph{dots}) for each event, and edges (called \emph{arrows}) that represent the activation relation of two events. Clinger's event diagram is typically drawn using parallel vertical swim-lanes for actors, where the dots are placed respecting their sequential execution order. Figure~\ref{fig::clinger} presents the Clingers' event diagram of an example actor model, shown in Listing~\ref{src::actor-model}. 

\begin{lstlisting}[language=rebeca, caption=An example of a simple actor model, label=src::actor-model]
reactiveclass AC1 {
  knownrebecs {AC2 ac2;}
  AC1() {
    self.msg1();
  }
  msgsrv msg1() {
    self.msg2();
    ac2.msg3();
  }
  msgsrv msg2() {
    self.msg1();
    ac2.msg4();
  }
}
reactiveclass AC2 {
  knownrebecs{AC1 ac1;}
  statevars{int sv;}
  AC2() {
    sv = 1;
  }
  msgsrv msg3() {
    ac1.msg1();
  }
  msgsrv msg4() {
    if (sv == 1)
      sv = 4;
    else
      sv = 3;
  }
}
main {
    AC1 ac1(ac2):();
    AC2 ac2(ac1):();
}
\end{lstlisting}

Clinger's event diagrams can be seen as the detailed representations of CDG. Intuitively, a CDG represents the possible activation relations of events derived from a static analysis of the model. Note that as actors are isolated, the only mechanism which may results in activating an event (causality among events) in an actor is sending a message to it. This way, the activation relation of events in a CDG can be extracted from the source codes of actor models by figuring out the message passing among actors. Using static analysis to find message passing results in over-approximatation of events activations in CDGs. Figure \ref{fig::cdg} illustrates the CDG which corresponds to the Clinder's event diagram of Figure \ref{fig::clinger}.

\begin{figure}
\centering
\subfigure[Clinger event diagram of an example actor model]{
\label{fig::clinger}
  \centering
  \small{
   \includegraphics[width=.18\textwidth]{resources/clinger.pdf}
  }
}
\qquad
\subfigure[CDG of an example actor model]{
\label{fig::cdg}
  \centering
  \small{
   \includegraphics[width=.18\textwidth]{resources/cdg.pdf}
  }
}
\caption{Clinger event diagram versus CDG of an example actor model.}
\label{fig::clinger-cdg}
\end{figure}

In \cite{DBLP:journals/eceasst/KhamespanahSMSR15} we designated and proved a relation between the cycles in the CDG and the cycles in state spaces. We devised a distribution policy for the distributed model checker of \emph{Rebeca} based on the CDG. The new distribution policy increases the efficiency of distributed model checking by increasing the locality of the accepting cycles. Experimental evidence supports that this new policy improves cycle locality, and decreases model checking time and memory in practice.