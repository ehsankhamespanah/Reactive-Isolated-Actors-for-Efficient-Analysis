\documentclass[conference]{IEEEtran}
\IEEEoverridecommandlockouts
% The preceding line is only needed to identify funding in the first footnote. If that is unneeded, please comment it out.
\usepackage{cite}
\usepackage{amsmath,amssymb,amsfonts}
\usepackage{algorithmic}
\usepackage{graphicx}
\usepackage{textcomp}
\usepackage{xcolor}
\def\BibTeX{{\rm B\kern-.05em{\sc i\kern-.025em b}\kern-.08em
    T\kern-.1667em\lower.7ex\hbox{E}\kern-.125emX}}
\begin{document}

\title{Analysing Real-time Reactive Systems using Timed Actors
\thanks{The authors would like to acknowledge DPAC and SEADA.}
}

\author{\IEEEauthorblockN{Marjan Sirjani}
\IEEEauthorblockA{\textit{IDT Department} \\
\textit{MDH}\\
Vasteras, Sweden \\
marjan.sirjani@mdh.se}
\and
\IEEEauthorblockN{Ehsan Khamespanah}
\IEEEauthorblockA{\textit{School of Computer Sceince} \\
\textit{Reykjavik University)}\\
City, Country \\
email address}
\and
\IEEEauthorblockN{Fatemeh Ghassemi}
\IEEEauthorblockA{\textit{School of ECE} \\
\textit{University of Tehran)}\\
City, Country \\
email address}
}

\maketitle

\begin{abstract}
I will introduce timed actors for modeling distributed systems and will explain our theories, techniques and tools for model checking and performance evaluation of such models. Timed Rebeca can be used to model asynchronous event-based components in systems, and real time constraints can be captured in the language. I will explain how floating-time transition system can be used for model checking of such models when we are interested in event-based properties, and how it helps in state space reduction. I will show different applications of our approach including analysing a wireless sensor network application, mobile ad-hoc network protocols, network-on-chip designs, and a macroscopic agent-based simulation of urban planning.
\end{abstract}

\begin{IEEEkeywords}
Actors, Real-time systems\end{IEEEkeywords}

\section{Introduction}

Message of the paper:
The actor-based language, Rebeca, provides a usable and analyzable model for distributed, concurrent, event-based asynchronous systems (Cyber-Physical systems).

Floating Time Transition System is a natural event-based semantics for timed actors, giving us a significant amount of reduction in the state space, using a non-trivial novel idea.

How models shape the thought and ease the analysis


Reactive systems

Faithflness:
From Rocco paper: a major challenge in designing languages is to devise appropriate abstractions and linguistic primitives to deal with the specificities of the domain under investigation.

Reactive actors: 

FTTS: Isolation

Hybrid Rebeca

wRebeca


\section*{Acknowledgment}
\section*{References}



\begin{thebibliography}{00}

\end{thebibliography}

\end{document}
